\documentclass[../main.tex]{subfiles}

\graphicspath{{\subfix{../images/}}}

\newpage


\begin{document}

%% BEGIN WRITING %%

\section{Conversione DAC e ADC}

Con questa sezione andiamo ad introdurre alcuni concetti fondamentali utili all'introduzione dei segnali numerici detti anche digitali. In particolare ci si soffermerà sul teorema del campionamento e sulla rispettiva conversione tra segnali analogici e digitali (Digital to Analog Conversion e Analog to Digital Conversion).

\subsection{Introduzione}

Fino ad ora ci siamo fermati a trattare segnali che come abbiamo visto fin dall'inizio erano segnali analogici ovvero continui sia sull'asse dei tempi che sull'asse delle ampiezze.


\subsection{Tipologie di segnali}


\subsection{Campionamento}


\subsection{Errore di quantizzazione}













\end{document}