\documentclass[../main.tex]{subfiles}

\graphicspath{{\subfix{../images/}}}

\newpage


\begin{document}

%% BEGIN WRITING %%


\section{Richiami di algebra lineare e geometria - ALG}

All'interno di questa sezione andremo a ripassare i concetti fondamentali dell'algebra e della geometria introdotti nel corso di Algebra Lineare e Geometria.

Questa sezione si colloca all'inizio del documento in quanto dovrebbe contenere nozioni già apprese dallo studente ma alla quale possiamo dare comunque uno sguardo per un veloce ripasso.
In particolare questa parte si colloca prima della trattazione dei segnali come vettori introdotti alla fine del terzo capitolo sulla trattazione dei segnali analogici. 

\subsection{Metrica}

Iniziamo il nostro ripasso andando a vedere come prima cosa la definizione di metrica riportata di seguito.

\newtheorem{theorem}{Definizione}[section]

\begin{theorem}[\textbf{Metrica}]\

Sia V un insieme (in particolare, uno spazio vettoriale reale o complesso).
Si dice metrica (o funzione distanza) su  V una funzione

\[ d : V \times V \rightarrow \mathbb{R} \]\

la quale associa a ogni coppia di punti (o vettori) \textbf{x},\textbf{y} $\in$ V un numero reale \textbf{d(x,y)} chiamato distanza tra x e y, tale che siano verificate le proprietà fondamentali.

\end{theorem}

Seguono le proprietà fondamentali della metrica:

\begin{enumerate}
	\item[\textbf{-}]\textbf{Non negatività:} \[d(\textbf{x},\textbf{y}) \geq 0 \;\; \forall \textbf{x},\textbf{y} \in V \]
	
	Questa proprietà conferma la non negatività dell'operazione. In altri termini, considerando la metrica come distanza, possiamo dire che non esistono distanze negative.
	
	\item[\textbf{-}]\textbf{Identità dell’indiscernibile:} \[ d(\textbf{x},\textbf{y}) = 0 \hspace{10pt} \Longleftrightarrow  \hspace{10pt} \textbf{x} = \textbf{y} \]
	
	Questa proprietà afferma che se la metrica tra due elementi di V è nulla allora i due elementi sono lo stesso. In altri termini, considerando la metrica come distanza, possiamo dire che un punto avente distanza nulla da un altro è il punto stesso.
	
	\item[\textbf{-}]\textbf{Simmetria:} \[ d(\textbf{x},\textbf{y}) = d(\textbf{y},\textbf{x}) \hspace{20pt} \forall \textbf{x},\textbf{y} \in V \]
	
	Questa proprietà afferma che la metrica è un'operazione simmetrica. Possiamo dire solo in termini informali che essa sia commutativa come le operazioni algebriche. Ragionando sempre in termini di distanza lo spazio che separa due punti è lo stesso che io lo misuri a partire dal primo o dal secondo.
	
	
	\item[\textbf{-}]\textbf{Disuguaglianza triangolare:} \[ d(\textbf{x},\textbf{z}) \;\;\leq \;\;d(\textbf{x},\textbf{y})\; + \;d(\textbf{y},\textbf{z})   \]
	
	
	Questa proprietà afferma che la distanza tra due punti sia sempre minore della somma delle distanze tra i due punti passando per un terzo punto che non si trova sulla direttrice dei primi due. A livello concettuale preserva l’idea che ogni percorso intermedio è almeno lungo quanto quello diretto e assicura la coerenza geometrica dello spazio metrico.
	
	
\end{enumerate}

Notiamo infine come la distanza Euclidea tra vettori soddisfa tutte le precedenti condizioni, ed è dunque una possibile metrica all'interno di uno spazio opportuno.

\subsection{Spazio vettoriale}

Passiamo ora oltre andando a definire uno spazio vettoriale (S.V.) dandone la definizione ed elencando le operazioni che è possibile effettuare su di esso.

\begin{theorem}[\textbf{Spazio vettoriale}]\

Sia $\mathbb{K}$ un campo (ad esempio $\mathbb{R}$ o $\mathbb{C}$). Uno spazio vettoriale su $\mathbb{K}$ è un insieme V i cui elementi si chiamano vettori, su cui sono definite due operazioni: somma vettoriale e moltiplicazione per scalare.
	
\end{theorem}

Seguono le operazioni definite su uno spazio vettoriale:

\begin{enumerate}
	\item[\textbf{-}]\textbf{Somma vettoriale (+) :} \[ V \times V \rightarrow V, \hspace{10pt} (\textbf{u},\textbf{v}) \rightarrow \textbf{u} + \textbf{v}     \]\
	\item[\textbf{-}]\textbf{Moltiplicazione per scalare ($\cdot$) :} \[ \mathbb{K} \times \rightarrow V, \hspace{10pt} (\alpha,\textbf{v}) \rightarrow \alpha \textbf{v}\]\
\end{enumerate}

Queste operazioni appena elencate devono soddisfare le seguenti otto proprietà fondamentali per ogni \textbf{u}, \textbf{v}, \textbf{w} $\in$ V e $\alpha$, $\beta$ $\in$ $\mathbb{K}$.
Sono riportate di seguito le proprietà fondamentali della somma vettoriale e del prodotto per scalare. Le prime quattro fanno riferimento alla somma vettoriale mentre le altre quattro si riferiscono al prodotto per scalare.

\begin{enumerate}
	\item[\textbf{-}]\textbf{Associatività:}
	\item[\textbf{-}]\textbf{Commutatività:}
	\item[\textbf{-}]\textbf{Elemento neutro:}
	\item[\textbf{-}]\textbf{Elemento opposto:}
	\item[\textbf{-}]\textbf{Associatività rispetto ai prodotti scalari:}
	\item[\textbf{-}]\textbf{Elemento neutro dello scalare:}
	\item[\textbf{-}]\textbf{Distributività rispetto alla somma di vettori:}
	\item[\textbf{-}]\textbf{Distributività rispetto alla somma di scalari:}
\end{enumerate}

\subsection{Norma e distanza}

Passiamo oltre ed andiamo ad introdurre ora i concetti di norma 




















\end{document}