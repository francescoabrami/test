\documentclass[../main.tex]{subfiles}

\graphicspath{{\subfix{../images/}}}

\newpage


\begin{document}

%% BEGIN WRITING %%

\section{Introduzione al corso}

\subsection{Segnali}

Iniziamo ora dando un piccolo sguardo a questo corso facendo dei piccoli esempi per avvicinarci, seppur in maniera approssimativa, agli argomenti che tratteremo.\\
In particolare questa sezione è pensata per introdurre diversi concetti, in modo del tutto approssimativo, per gettare delle basi di quello che verrà trattato in seguito all'interno di questo documento. I concetti all'interno di questa sezione potranno sembrare buttati a caso o addirittura sopraffare il lettore ma ricordiamo che questa non è altro che un'introduzione e che tutti i concetti verranno trattati in modo esaustivo nel resto dell'elaborato.\\

Iniziamo dando la definizione di \textbf{segnale} che in maniera del tutto generale può essere definito come \textit{"una funzione reale o complessa nella variabile tempo"} ed in particolare possiamo dire che l’informazione trasportata da un segnale è contenuta nella \textit{“forma”} del segnale stesso ovvero nella sua evoluzione nel tempo.

\subsubsection{Tipologie di segnali}

In particolare possiamo fin da subito fare degli esempi di alcune tipologie di segnali che sono rilevanti dal punto di vista ingegneristico. Tra queste tipologie troviamo:

\begin{enumerate}
	\item[-] Segnale \textbf{elettrico}
	\item[-] Segnale \textbf{vocale}
	\item[-] Segnale \textbf{audio}
	\item[-] Segnale \textbf{dati}
\end{enumerate}

In particolare partiamo fin da subito prendendo in esempio un segnale elettrico che nella maggior parte dei casi pratici è il segnale maggiormente utilizzato per la propagazione dell'informazione che essi trasportano ma che non per forza sia il segnale che originariamente avevamo preso in analisi.\\
In particolare ci è possibile passare da una tipologia di segnale ad un'altra attraverso dei componenti detti \textbf{trasduttori} ossia dispositivi che
convertono una qualsiasi grandezza scalare in un segnale elettrico indipendentemente dalla loro natura (diretto, variabile, sinusoidale o pulsante). Si utilizza principalmente il segnale elettrico come output dei trasduttori in quanto possiamo maneggiarlo in modo molto più semplice e comodo rispetto ad altre forme.\\
Evidenziamo ora alcune tipologie di trasduttori tra cui troviamo:

\begin{enumerate}
	\item[-]\textbf{ Pressione sonora }(audio, microfono)
	\item[-]\textbf{Intensità luminosa} (immagini e video)
	\item[-] \textbf{Velocità}, \textbf{temperatura}, etc.
\end{enumerate}

\subsection{Possibili operazioni sui segnali}

Introdotto ora il concetto di segnale e di trasduttore andiamo a vedere alcune possibili operazioni che possiamo effettuare sul segnale che stiamo analizzando.
In particolare ci è possibile effettuare principalmente operazioni di:

\begin{enumerate}
	\item[\textbf{-}] \textbf{Elaborazione:} questa operazione ci permette di raggiungere vari obiettivi che possono essere: l'eliminazione di rumore, l'estrazione di componenti più rilevanti per le successive elaborazioni oppure la capacità di filtrare componenti spurie generate dai trasduttori.
	\item[\textbf{-}] \textbf{Trasmissione:} con questa operazione ci è possibile trasportare l’informazione contenuta nel segnale per una certa distanza, al fine di essere fruita in un posto diverso da quello dove è stata generata. Questa operazione è al centro dello studio delle telecomunicazioni.
	\item[\textbf{-}] \textbf{Memorizzazione:} capacità di conservare l’informazione contenuta nel segnale rendendola fruibile anche a distanza di tempo.
\end{enumerate}













\subsection{Segnali analogici digitali e la loro trasmissione}

Prima di proseguire oltre andiamo a dare una doverosa definizione delle differenti tipologie di segnale che andremo ad analizzare.
In particolare possiamo individuare due tipologie fondamentali di segnali:

\begin{enumerate}
	\item[\textbf{-}] \textbf{Analogici:} Rientrano in questa categoria tutti quei segnali che sono continui nel tempo e nella loro ampiezza ovvero possono assumere qualsiasi valore in un qualsiasi istante di tempo. Per fare degli esempi una sinusoide è un segnale analogico in quanto è infinitamente continua nel tempo ma limitata in ampiezza. In particolare un segnale analogico in quanto continuo nel tempo ammette anche variazioni istantanee in ampiezza anche infinite in quanto ammesse dalla sua stessa definizione.
	
	\item[\textbf{-}] \textbf{Digitali o Numerici:} Rientrano invece in questa categoria tutti quei segnali che assumo valori discreti nel tempo, ovvero non sono più continui, mentre in ampiezza possono assumere un numero finito di valori.	
\end{enumerate}

Vedremo inoltre come esista una categoria di segnali a tempo discreto ma continui in ampiezza che verranno approfonditi nella seconda parte del corso. In particolare questi segnali sono fondamentali nell’ambito dell’elaborazione numerica degli stessi.


\subsubsection{Scomparsa dei sistemi analogici}



\subsection{Conversione analogico digitale - ADC}


\subsection{Possibili Applicazioni dell’Analisi ed Elaborazione dei Segnali}

Concludiamo ora questa sezione andando ad elencare velocemente quali possono essere delle possibili applicazioni dell'analisi ed elaborazione dei segnali.



















\end{document}
