\documentclass[../main.tex]{subfiles}

\graphicspath{{\subfix{../images/}}}

\newpage


\begin{document}

%% BEGIN WRITING %%

\section{Sistemi}

Viste le sezioni precedente ed introdotte le proprietà della trasformata di Fourier possiamo ora affrontare il prossimo argomento ovvero i sistemi lineari (LS - Linear Systems) di cui elencheremo definizione, proprietà e caratteristiche. Come vedremo più avanti la nostra attenzione ricadrà per la maggiorparte su una sottocategoria dei sistemi detti sistemi lineare tempo-invarianti.

\subsection{Introduzione ai sistemi}

Vista la parte precedente possiamo ora dare la definizione di sistema. In particolare un sistema nell'ambito della teoria dei segnali è un qualsiasi elemento che trasforma un segnale dato in ingresso in un'altro segnale che verrà prodotto in uscita.

Di seguito possiamo vedere una possibile rappresentazione di un sistema attraverso una rappresentazione dove il sistema è un singolo blocco. In particolare il sistema applica una generica trasformazione $T$ al segnale in ingresso generandone un'altro in uscita.







Data la definizione di sistema possiamo ora elencarne alcuni per fornire degli esempi pratici legati al mondo reale.
Sono un'esempi di sistema elementi come:

\begin{enumerate}
	\item[\textbf{-}] \textbf{Amplificatori Elettronici:} Questi dispositivi sono in grado di moltiplicare il segnale in ingresso per un certo valore riproducendo in uscita lo stesso segnale ora amplificato di un certo valore.
	\item[\textbf{-}] \textbf{Filtri Analogici:} Questi dispositivi sono in grado di filtrare alcune frequenze di un segnale andando a riprodurre in uscita solo una porzione delle spettro in frequenza del segnale posto in ingresso. 
	\item[\textbf{-}] \textbf{Sistemi di trasmissione su lunga distanza:} Anche le trasmissioni a lunghe distanze introducono delle alterazioni sui segnali che quindi possono essere modellati come delle particolari tipologie di sistema.
	\item[\textbf{-}] \textbf{Trasduttori Fisici:} Sono dispositivi in grado di generare, solitamente una tensione o corrente, in base all'ingresso (temperatura, pressione etc...) che viene applicata ad essi.  
	\item[\textbf{-}] \textbf{Sistemi di registrazione e successiva riproduzione:} Anche la registrazione di un'immagine o di un suono passano attraverso un sistema che le memorizzi per poi essere riprodotte dopo un certo periodo di tempo.
\end{enumerate}

Concludiamo ora questa prima sottosezione andando a fornire una classificazione delle varie tipologie di sistemi che potremo incontrare:

\begin{enumerate}
	\item[\textbf{-}] \textbf{Lineari e non Lineari}: Con questa tipologia di sistemi facciamo riferimento ad un sistema che rispetta i principi di sovrapposizione e omogeneità ovvero che applicano una trasformazione di tipo lineare come somma, sottrazione etc. Sono esclusi da questa categoria tutti i sistemi che non rispettano tali regole. Per esempio un sistema che produce il quadrato del segnale in ingresso non è detto lineare.
	
	\item[\textbf{-}] \textbf{Senza memoria e con memoria}: I sistemi senza memoria (m
	emoryless) sono rappresentati dall'insieme di tutti quei sistemi dove l'uscita dipende solo dal valore istantaneo del segnale in ingresso e non conta la storia passata del segnale. Nei sistemi con memoria l'uscita dipende dal valore passato del segnale e non da quello istantaneamente assunto in quel momento.
	
	\item[\textbf{-}] \textbf{Tempo varianti e tempo invarianti}: Queste tipologie di sistemi sono tali che le caratteristiche dello stesso non cambino nel tempo. Per esempio se un sistema applica un ritardo costante al segnale in ingresso è tempo invariante non lo è un sistema dove le sue caratteristiche (per esempio il valore di amplificazione) cambiano in funzione del tempo.

	\item[\textbf{-}] \textbf{Causali e non causali}: Sono sistemi che rispettano il principio di causa effetto. In questi sistemi l'uscita dipende solo dal presente o dal passato del segnale e mai dal futuro. Per i sistemi non causali questa condizione non viene rispettata. Vediamo subito che tutti i sistemi fisici sono esclusivamente causali.
	
	\item[\textbf{-}] \textbf{Reali e non reali}: Questa tipologia di sistemi lavora con grandezze reali come tensioni e correnti. Un sistema non reale utilizza valori complessi composti da una parte reale ed una immaginaria.
	
	\item[\textbf{-}] \textbf{Stabili e non stabili}: Questi sistemi seguono il principio BIBO (Bounded Input Bounded Output) ovvero che ad un ingresso limitato in ampiezza, energia e potenza equivale un'uscita anch'essa limitata. I sistemi non stabili non rispettano questa regola portando alla produzione di segnali in uscita con oscillazioni infinite. 
		
\end{enumerate}

\subsection{Classificazione di sistemi}

Introdotte le principali classificazioni dei sistemi possiamo ora andare a vedere nel dettaglio alcune di esse.

\subsubsection{Sistemi lineari}
\subsubsection{Sistemi tempo invarianti}
\subsubsection{Sistemi senza memoria}
\subsubsection{Sistemi senza memoria e tempo invarianti}


\subsection{Prodotto di convoluzione}
\subsubsection{Costruzione grafica}
\subsubsection{Convoluzione di due porte}
\subsubsection{Proprietà del prodotto di convoluzione}

\section{Sistemi LTI}

\subsection{Risposta all'impulso}
\subsection{Funzione di trasferimento}

\subsection{Tipologie di filtri realizzabili}






















\end{document}