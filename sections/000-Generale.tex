\documentclass[../main.tex]{subfiles}

\graphicspath{{\subfix{../images/}}}

\newpage


\begin{document}

%% BEGIN WRITING %%

\section{Introduzione Generale}

\subsection{Suddivisione Corso}

Il corso è strutturato come molti altri evidenziando tre diversi momenti lezioni, esercitazioni ed homework. Per quanto riguarda il materiale trattato fare riferimento all’indice di questo documento.

\subsection{Materiale}

Questo elaborato si basa principalmente sul materiale fornito durante il corso, tra cui slide, esercizi, temi d'esami ed homework. L'elaborato segue il programma del corso nel modo più fedele possibile anche se l'ordine con cui il materiale viene trattato potrebbe non corrispondere allo svolgimento delle lezioni tenute in aula.
 Infine si sottolinea la presenza di vari approfondimenti sugli argomenti indicati come extra, che richiedono uno studio autonomo da parte dello studente. Queste informazioni sono state tratte in parte dai testi consigliati per il corso: \\
 
 % TODO Finire libri per info extra
 
Nell'eventualità in cui il lettore in possesso di questo elaborato abbia individuato degli errori, inesattezze o voglia proporre delle migliorie non esiti a contattare il seguente indirizzo di posta elettronica: \href{mailto:francesco.abrami@studenti.polito.it?subject=Teoria ed Elaborazione dei Segnali - 02MOOOA&body= Buongiorno, 
scrivo riguardo al documento LaTex del corso di cui in oggetto.}{francesco.abrami@studenti.polito.it}

\subsection{Lezioni Esercitazioni ed Homework}

Lezioni, esercitazioni ed homewrok si svolgono nelle stesse aule.
Per gli homework é richiesta della preparazione preliminare al fine di svolgere correttamente tutte le attività utilizzando lo slot per verificare quanto fatto. Le sezioni finali di questo elaborato propongono riassunte le spiegazioni delle esercitazioni e degli homework affrontati durante il corso.
Ricordiamo che gli homework, ovvero esercitazioni al computer, svolti con l'utilizzo di Matlab o Python sono facoltativi e possono portare fino ad un massimo di due punti al punteggio ottenuto all'esame.\\

Infine segue ora un velocissimo elenco delle abilità che verranno acquisite all'interno del corso:

\begin{enumerate}
	\item[-] Classificazioni dei segnali.
	\item[-] Tecniche di analisi in frequenza dei segnali a tempo continuo.
	\item[-] Sistemi lineari tempo-invarianti (LTI), e la loro rappresentazione nel dominio del tempo e della frequenza.
	\item[-] Tipologie fondamentali di filtri lineari.
	\item[-] Processi stocastici (casuali) e loro rappresentazione spettrale.
	\item[-] Tecniche per il passaggio da segnali a tempo continuo ai segnali a tempo discreto, e viceversa.
	\item[-] Tecniche per l’elaborazione in frequenza dei segnali a tempo discreto.
	\item[-] Tecniche per l'analisi dei sistemi LTI a tempo discreto e trasformata Z.
	\item[-] Tecniche di filtraggio numerico, tipologie di filtri numerici (FIR, IIR).
\end{enumerate}

\subsection{Prova d'Esame}

L’esame finale si basa su uno scritto obbligatorio, costituito da domande ed esercizi a risposta multipla) e un orale opzionale (a discrezione dello studente o del docente).\\
In particolare il tempo a disposizione dello studente per la prova scritta (su
piattaforma Exam o supporto cartaceo) è di 90 minuti e non è possibile consultare materiale didattico né appunti o altri testi. È consentito utilizzare esclusivamente la calcolatrice e un formulario fornito dal docente.\\
Infine esiste la possibilità di sostenere un'esame orale alla quale per essere ammessi occorre ottenere un voto alla prova scritta superiore o uguale a 15/30. L'orale ha una durata di circa 15 minuti, e riguarda tutti gli argomenti trattati nelle lezioni e nelle esercitazioni.\\

Ricordiamo infine che esempi di esami scritti degli anni precedenti saranno messi a disposizione attraverso il portale della didattica.

\end{document}
