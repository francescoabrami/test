\documentclass[../main.tex]{subfiles}

\graphicspath{{\subfix{../images/}}}

\newpage


\begin{document}

%% BEGIN WRITING %%

\section{Sequenze fondamentali}

Vista la sezione precedente dove abbiamo introdotto i segnali a tempo discreto andiamo ora a vedere delle sequenze, ovvero dei segnali a tempo discreto, che troveremo ed utilizzeremo spesso all'interno di questi prossimi capitoli.
In particolare, come potremo osservare, queste sequenze fondamentali non sono altro che il risultato ottenuto dalla trasformazione a tempo discreto degli stessi segnali a tempo continuo visti in precedenza. Ricordiamo infine come l'operazione appena descritta non sia affatto un'operazione matematica ma piuttosto un modo di pensare questi segnali: vedremo come ci saranno opportune regole ed accorgimenti da apportare al ragionamento proposto.

\subsection{Sequenza gradino unitario}

Introdotto il ragionamento generale possiamo ora passare a vedere la prima sequenza rappresentata dal gradino unitario.
In particolare il segnale si definisce in tempo discreto nel seguente modo:

\[ u(n) = \begin{cases}
	0, \hspace{10pt} n < 0\\
	1, \hspace{10pt} n \geq 0
\end{cases} \]\

Riportiamo ora il grafico del segnale generato con MATLAB:

% TODO GRAFICO GRADINO UNITARIO MATLAB

Visto che il grafico è generato attraverso del codice riteniamo opportuno riportare, almeno per questa sezione di corso, anche il codice che produce i grafici riportati all'interno di questo documento.
Segue ora il codice MATLAB utilizzato per la generazione del segnale gradino unitario:\\

\lstinputlisting[language=Matlab, firstline = 3, lastline = 12]{code/TESGRAPHS.m}


\subsection{Delta di Kroenecher}

Passiamo oltre ed andiamo a vedere come il segnale delta di Dirac viene interpretato in tempo discreto. Prima di tutto, per non confonderci con le due tipologie, utilizzeremo il termine delta per indicare quello a tempo continuo mentre useremo delta di Kroenecher per indicare il segnale a tempo discreto.
Questa tipologia di delta è rappresentata da un impulso unitario, centrato nell'origine, non più di ampiezza infinita ma avente ampiezza pari ad uno.

Possiamo scrivere quanto appena descritto nel seguente modo:

\[ u(n) = \begin{cases}
 0, \hspace{10pt} n \neq 0 \\
 1, \hspace{10pt} n = 0
\end{cases}
\]\

Ricordiamo infine che la definizione della delta di Dirac $\delta(t)$ è analoga a quella appena vista ma su un supporto continuo dove la funzione assume un valore infinito nell'origine.

\[ \delta(t) = \begin{cases}
 0, \hspace{15pt} t \neq 0 \\
 \infty, \hspace{10pt} t = 0
\end{cases}
\]\

Segue ora il grafico del segnale:

% TODO GRAFICO DELTA UNITARIO MATLAB

Come nel caso precedente riportiamo anche il codice MATLAB utilizzato per la generazione del segnale delta di Kroenecher:\\

\lstinputlisting[language=Matlab, firstline = 16, lastline = 25]{code/TESGRAPHS.m}

Infine prima di passare oltre possiamo notare come, a differenza dei segnali a tempo continuo, il gradino unitario e la delta di Kroenecher non creino alcuna criticità caratteristica dei corrispondenti segnali a tempo continuo.\\

Riportiamo infine una proprietà della delta di Kroenecher.

\begin{prop*}[\textbf{Segnale come somma di impulsi}]\

\[x(n) = \sum_{i\;=\;-\infty}^{+\infty} x(i) \delta(n-1)\]\
	
\end{prop*}

Questa proprietà fondamentale della delta numerica indica la possibilità di esprimere ogni segnale $x(n)$ come somma di impulsi secondo la relazione vista sopra dove il termine $\delta(n-i)$ è la delta di Kroenecher centrata nell’istante di tempo $i$.
Ci è possibile verificare che:

\[ x(n) \delta(n) = x(0)\delta(n) \]
\[ x(n) \delta(n-i) = x(i)\delta(n-1) \]\

Infine possiamo andare ad esplicitare la relazione tra la delta numerica ed il gradino unitario. Ci è infatti possibile scrivere $\delta(n)$ come la differenza tra due gradini traslati nel seguente modo:

\[ u(n) = \sum_{i = 0}^{+\infty} \delta(n-i) = \delta(n) + \delta(n+1) + \delta(n+2) \dots \]
\[ \delta(n) = u(n) - u(n-1) \]\

Possiamo notare infine attraverso la differenza di sequenze gradino, aventi diversi supporti, ci è possible creare delle funzioni porta nel dominio discreto.

\subsection{Sequenza sinc}

Proseguendo il nostro percorso incontriamo la sequenza sinc($n$) che andiamo a definire nel seguente modo dove $N$ è un numero intero positivo:

\[ \unit{sinc}\bigg(\frac{n}{N}\bigg) =  \frac{\sin\big(\pi \frac{n}{N}\big)}{\pi \frac{n}{N}}\]\

Vista ora la definizione della sequenza andiamo a riportarne il grafico.

% TODO GRAFICO SINC

Come per gli esempi fatti in precedenza segue ora il codice MATLAB utilizzato per generare il grafico riportato sopra, dove abbiamo utilizzato un valore $y$ pari a $n/3$:

\lstinputlisting[language=Matlab, firstline = 29, lastline = 37]{code/TESGRAPHS.m}

Infine, per quanto riguarda la sequenza $sinc$() riportiamo anche alcuni altri grafici ottenuti andando a cambiare il valore di $y$ prima con $n/2$ e successivamente con $n$. 

% TODO GRAFICI SINC(n)

Come possiamo notare i grafici differiscono molto l'uno dall'altro nonostante si stia utilizzando sempre la stessa funzione per generarli. Capiremo in un esempio più avanti che cosa stia causando questo problema e cosa fare per evitarlo.

\subsection{Sequenza porta}

Proseguiamo ora andando ad analizzare la sequenza porta che indichiamo nel seguente modo:

\[ p_{2k+1}[n] = \begin{cases}
	1 \hspace{10pt} |n| \leq k \\
	0 \hspace{10pt} |n| > k 
\end{cases} \]\

La definizione appena vista indica una porta di supporto pari a $2k + 1$ centrata in un punto. Notiamo come il supporto sia dato da $2k + 1$ in quanto dobbiamo considerare il punto in cui è centrata la porta più un numero pari di $k$ punti da una parte e dall'altra per ottenere la porta. Ricordiamo che nel dominio del tempo discreto il supporto di una sequenza è dato dal numero di punti tra gli estremi composti da punti non nulli. In altre parole è dato dall'intervallo più grande che si possa ottenere considerando come estremi punti non nulli con la possibilità di avere punti nulli al suo interno.\\

Come fatto in precedenza andiamo a rivedere la definizione di porta nel dominio continuo del tempo:


\[ p_{T}(t) = \begin{cases}
	1 \hspace{10pt} t \in [-\tfrac{T}{2}, + \tfrac{T}{2}]\\
	0 \hspace{10pt} \unit{altrove}
\end{cases} \]\

Ora vista la sua definizione andiamo a riportarne un grafico ottenuto in MATLAB.
In questo caso si è scelto di rappresentare una porta di supporto % TODO SUPPORTO

% TODO GRAFICO PORTA

Come per gli altri esempi si riporta di seguito il codice utilizzato per la generazione del grafico appena visto:

\lstinputlisting[language=Matlab, firstline = 41, lastline = 50]{code/TESGRAPHS.m}

\subsection{Sequenza triangolo}

Passiamo ora invece ad analizzare la sequenza triangolare dandone fin da subito una definizione:

\[ \unit{tri}(n) = \begin{cases}
	1 - \tfrac{|n|}{N}, \hspace{10pt} |n| \geq N \\
	0\;, \hspace{10pt} |n| < N
 \end{cases} \]\

In particolare in base al valore scelto di $N$, che ricordiamo essere un numero intero positivo, si otterrà una sequenza triangolare avente supporto pari a $2N+1$.
Come per i casi precedenti ricordiamo anche qui la definizione del corrispondente segnale a tempo continuo:

\[ \unit{tri}(t) = \begin{cases}
	1 - \tfrac{|t|}{T}, \hspace{10pt} |t| \geq T \\
	0\;, \hspace{10pt} |t| < T	
\end{cases} \]\

Ricordiamo che in questo caso $t$, ovvero la variabile tempo è continua e non discreta come succede per $n$.

Riportiamo di seguito il grafico della sequenza triangolare ottenuta in MATLAB:

% TODO GRAFICO MATLAB

Riportiamo anche di seguito il codice utilizzato per generare il grafico appena visto:

\lstinputlisting[language=Matlab, firstline = 54, lastline = 63]{code/TESGRAPHS.m}

\subsection{Sequenza esponenziale}

Concludiamo ora la lista delle sequenze fondamentali andando a prendere in analisi la sequenza esponenziale decrescente.
Andiamo a definire la sequenza attraverso questa formula:

\[ x(n) = a^n \cdot u(n) \]\

Dove, in generale, si assume che $a$ sia un numero complesso e che se $a$ è reale la funzione risulta avere segno costante nel caso in cui il valore di $a$ sia strettamente maggiore di zero. Nel caso opposto la sequenza risulta essere a segni alterni.

Anche in questo caso riportiamo la definizione del segnale corrispondente a questa sequenza nel dominio discreto del tempo:

\[ x(t) = e^{-\alpha t} \cdot u(t) \]\

Ricordiamo che in questo caso stiamo considerando funzioni esponenziali decrescenti come anche succede per quelle a tempo discreto. Nel caso a tempo continuo si assume che il valore dell'esponente sia strettamente minore di zero ovvero $\alpha$ deve assumere un valore strettamente positivo. Per quanto riguarda il dominio del tempo discreto si assume che la base dell'esponenziale, ovvero $a$ sia compresa tra zero ed uno dove entrambi gli estremi sono esclusi.\\

Riportiamo ora una coppia di grafici in cui le sequenze rappresentate hanno lo stesso valore assoluto di $a$ ma diverso segno.
% TODO COPPIA EXP SEQ.

Infine riportiamo anche il codice utilizzato per generare il grafico della sequenza sempre positiva; per ottenere quella con segno alterno basta cambiare segno al valore di $a$ alla seconda riga.

\lstinputlisting[language=Matlab, firstline = 67, lastline = 76]{code/TESGRAPHS.m}

Concludiamo questa parte sulle sequenze esponenziali andando ad analizzare il caso in cui si utilizzi un $a$ avente valore complesso.
Nel caso in cui $a$ assuma un valore complesso, ovvero sia pari a $Ae^{j\theta}$ si otterrà che:

\[x(n) = \big( Ae^{j\theta} \big) \cdot u(n) = A^n \cdot e^{jn \theta} \cdot u(n) \]\


\subsubsection{Sinusoidi ed esponenziali a tempo discreto}

Vista ora la

\subsubsection{Esempi}
\subsubsection{Proprietà}



\subsection{Energia e potenza media}








\end{document}