\documentclass[../main.tex]{subfiles}

\graphicspath{{\subfix{../images/}}}

\newpage

%% ADD SECTION FRO CUSTOM COSE DISPLAY %%

\definecolor{codegreen}{rgb}{0,0.6,0}
\definecolor{codegray}{rgb}{0.5,0.5,0.5}
\definecolor{codepurple}{rgb}{0.58,0,0.82}
\definecolor{backcolour}{rgb}{1,1,1}

\lstdefinestyle{mystyle}{
    backgroundcolor=\color{backcolour},   
    commentstyle=\color{codegreen},
    keywordstyle=\color{magenta},
    numberstyle=\tiny\color{codegray},
    stringstyle=\color{codepurple},
    basicstyle=\ttfamily\footnotesize,
    breakatwhitespace=false,         
    breaklines=true,                 
    captionpos=b,                    
    keepspaces=true,                 
    numbers=left,                    
    numbersep=5pt,                  
    showspaces=false,                
    showstringspaces=false,
    showtabs=true,                  
    tabsize=2
}

\lstset{style=mystyle}

%% END SECTION FRO CUSTOM COSE DISPLAY %%


\begin{document}

%% BEGIN WRITING %%

\section{Segnali a tempo discreto}

Procediamo ora con il nostro percorso andando ad analizzare una nuova tipologia di segnali, e la loro elaborazione, attraverso metodi simili a quelli già visti fino ad ora.

\subsection{Introduzione}

Come appena detto stiamo per introdurre una nuova tipologia di segnali andando anche a trattarne la loro elaborazione.
In particolare la tipologia di segnali che tratteremo d'ora in poi sono detti a tempo discreti ovvero segnali la cui definizione non è continua lungo l'asse del tempo ma è limitata ad un numero discreto (in alcuni casi finito) di punti.
Procediamo ora ad una breve storia dell'elaborazione dei segnali numerici (ENS) per poi vederne delle possibili applicazione. Successivamente andremo ad definire meglio, attraverso definizioni proprietà ed operazioni, che cosa intendiamo per segnale numerico andando a fare le opportune classificazioni.

\subsubsection{Storia dell'ENS}

\subsubsection{Applicazioni dell'ENS}

\subsection{Segnali a tempo discreto}

\subsubsection{Durata}
\subsubsection{Causalità}
\subsubsection{Parità}
\subsubsection{Periodicità}
\subsubsection{Sequenze limitate in ampiezza}
\subsubsection{Sequenze sommabili}

\subsection{Operazioni}

\subsubsection{Somma, differenza, prodotto}
\subsubsection{Traslazione e ribaltamento}
\subsubsection{Scalamento temporale}













\end{document}